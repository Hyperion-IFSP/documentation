\begin{quadro}
\caption{Requisitos Não-Funcionais do MyMed}
\begin{tabularx}{\textwidth}{|l|l|X|}
\hline
\textbf{Código} & \textbf{Categoria} & \textbf{Descrição} \\ \hline
RNF01 & Segurança & O sistema deve assegurar a criptografia de dados sensíveis, como senhas e informações médicas. \\ \hline
RNF02 & Acesso & O sistema deve implementar controle de acesso baseado em perfis de usuário, como administradores ou cuidadores. \\ \hline
RNF03 & Sessão & O sistema deve realizar a validação de sessão com expiração automática após período de inatividade. \\ \hline
RNF04 & Armazenamento & O sistema deve armazenar os dados de forma segura, em conformidade com a Lei Geral de Proteção de Dados (LGPD). \\ \hline
RNF05 & Backup & O sistema deve executar backups regulares e automáticos para possibilitar a recuperação de dados em caso de falha. \\ \hline
RNF06 & Desempenho & O sistema deve responder às requisições em até, no máximo 1,5 segundos durante as operações. \\ \hline
RNF07 & Interface & O sistema deve realizar paginação e carregamento sob demanda (lazy loading) em listas extensas, como consultas e pacientes. \\ \hline
RNF08 & Interface & O sistema deve possuir uma interface intuitiva e acessível para usuários sem conhecimento técnico. \\ \hline
RNF09 & Compatibilidade & O sistema deve ser compatível com dispositivos móveis, mantendo todas as funcionalidades acessíveis. \\ \hline
RNF10 & Disponibilidade & O sistema deve permanecer disponível pelo menos 99,5\% do tempo durante o horário de operação. \\ \hline
\end{tabularx}
\fonte{Autores.}
\end{quadro}
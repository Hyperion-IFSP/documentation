Os requisitos não funcionais do sistema são apresentados a seguir, servindo de descrição das principais características que o sistema deve atender.

\begin{quadro}
    \caption{\label{quadro_requisitos_nf}Requisitos Não Funcionais}
    \begin{tabular}{|c|c|p{10cm}|}
        \hline
        \textbf{Código} & \textbf{Categoria} & \textbf{Descrição} \\ \hline
        RNF01  & Segurança       & Criptografia de dados sensíveis: como senhas e informações médicas \\ \hline
        RNF02  & Acesso          & Controle de acesso baseado em perfis de usuário: admin ou cuidador \\ \hline
        RNF03  & Sessão          & Validação de sessão com expiração automática por inatividade \\ \hline
        RNF04  & Backup          & Backups regulares e automáticos: para recuperação de dados \\ \hline
        RNF05  & Desempenho      & O sistema deve responder às requisições em até 5 segundos nas operações \\ \hline
        RNF06  & Concorrência    & Suportar múltiplos acessos simultâneos sem perda de desempenho \\ \hline
        RNF07  & Interface       & Interface intuitiva e acessível para usuários não técnicos \\ \hline
        RNF08  & Design          & Uso de padrões de design consistentes e amigáveis \\ \hline
        RNF09  & Manutenção      & O código-fonte deve ser modular e documentado, facilitando a manutenção \\ \hline
        RNF10  & Arquitetura     & Uso de arquitetura escalável \\ \hline
        RNF11  & Disponibilidade & O sistema deve estar disponível pelo menos 99,5\% do tempo \\ \hline
        RNF12  & Integridade     & Deve garantir integridade dos dados em operações simultâneas \\ \hline
        RNF13  & Recuperação     & Deve ser capaz de recuperar-se de falhas sem perda de dados \\ \hline
        RNF15  & Ambientes       & Ambientes separados para produção, testes e homologação \\ \hline
        RNF16  & Segurança       & O sistema deve exigir autenticação de usuário para qualquer operação de inserção, alteração ou exclusão de dados \\ \hline
    \end{tabular}
    \fonte{Autores.}
\end{quadro}

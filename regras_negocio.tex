<<<<<<< HEAD
\begin{quadro}
    \caption{\label{quadro_regras_negocio}Regras de Negócio}
    \begin{tabular}{|c|c|p{10cm}|}
        \hline
        \textbf{Código} & \textbf{Categoria} & \textbf{Descrição} \\ \hline
        RN01   & Cadastro                       & Apenas usuários com permissão de cuidador ou administrador podem cadastrar dependentes. \\ \hline
        RN02   & Cadastro                       & Cada dependente deve ter um CPF único e válido no sistema. \\ \hline
        RN03   & Cadastro                       & O sistema deve impedir o cadastro de dependentes com dados incompletos obrigatórios (ex: nome, CPF, data de nascimento, contato). \\ \hline
        RN04   & Cadastro                       & A exclusão de um dependente não deve arquivar todo o histórico, porém não há remoção definitiva do banco de dados. \\ \hline
        RN05   & Cadastro                       & A edição de dados sensíveis (como CPF) deve ser registrada em log de auditoria. \\ \hline
        RN06   & Consultas                     & Consultas só podem ser agendadas para dependentes cadastrados no sistema. \\ \hline
        RN07   & Consultas                     & O sistema deve impedir agendamento de duas consultas no mesmo horário para o mesmo dependente. \\ \hline
        RN08   & Consultas                     & Consultas só podem ser editadas ou canceladas antes da data e hora marcadas. \\ \hline
        RN09   & Logs                          & A edição ou exclusão de uma consulta criada deve ser registrada em log. \\ \hline
        RN10   & Tratamentos                   & Cada tratamento deve estar vinculado a um dependente específico e seu diagnóstico. \\ \hline
        RN11   & Tratamentos                   & A evolução de um tratamento deve ser registrada com data, cuidador responsável e observações. \\ \hline
        RN12   & Tratamentos                   & Prescrições médicas devem ser anexadas com extensão válida (PDF, JPG, PNG, etc) e tamanho máximo pré-definido. \\ \hline
        RN13   & Tratamentos                   & Um tratamento só pode ser concluído ou cancelado por cuidadores com permissão e deve ser registrado em log. \\ \hline
        RN14   & Autenticação                  & Todos os usuários devem possuir credenciais únicas (login e senha criptografados). \\ \hline
        RN15   & Autenticação                  & O acesso ao sistema deve ser restrito por perfis: administrador/cuidador. \\ \hline
        RN16   & Autenticação                  & Registros sensíveis só podem ser manuseados por perfis autorizados. \\ \hline
        RN17   & Manuseio e exportação de dados & Toda ação crítica (edição, exclusão, exportação) deve ser registrada com data, hora, usuário e tipo de operação. \\ \hline
        RN18   & Manuseio e exportação de dados & O sistema deve permitir pesquisas por múltiplos critérios combinados nas filtragens. \\ \hline
        RN19   & Manuseio e exportação de dados & Os relatórios devem permitir filtros por dependente, período e/ou tratamento. \\ \hline
        RN20   & Interface                    & Toda ação do sistema deve retornar mensagem clara de sucesso ou erro. \\ \hline
        RN21   & Interface                    & Listagens com mais de 20 itens devem utilizar paginação ou lazy loading. \\ \hline
        RN22   & Interface                    & O sistema deve ser acessível e compatível com dispositivos móveis. \\ \hline
        RN23   & Desenvolvimento e Manutenção & O código deve ser modular e seguir boas práticas de engenharia de software. \\ \hline
        RN24   & Desenvolvimento e Manutenção & Toda funcionalidade deve prever testes automatizados (testes unitários). \\ \hline
    \end{tabular}
    \fonte{Autores.}
=======
As regras de negócio são fundamentais para garantir o correto funcionamento do sistema, assegurando que as operações atendam aos requisitos legais e funcionais. A seguir, apresentamos uma tabela com as principais regras de negócio definidas para o sistema de gestão de dependentes.

\begin{quadro}
\caption{Regras de Negócio do MyMed}
\begin{tabularx}{\textwidth}{|l|l|X|}
\hline
\textbf{Código} & \textbf{Categoria} & \textbf{Descrição} \\ \hline
RN01 & Cadastro & Apenas usuários com permissão de cuidador ou administrador podem cadastrar pacientes. \\ \hline
RN02 & Segurança & A exclusão de qualquer registro relacionado a um usuário, como o perfil de um paciente ou seu próprio perfil, deve arquivar todo o histórico sem remoção definitiva do banco de dados durante 15 dias, conforme prazo de resposta da LGPD. \\ \hline
RN03 & Tratamentos & Um plano de tratamento só pode ser criado se houver, no momento do registro, ao menos um diagnóstico relacionado. \\ \hline
RN04 & Tratamentos & A conclusão ou o cancelamento de um tratamento só podem ser realizados por cuidadores com permissão e devem ser registrados em log. \\ \hline
RN05 & Tratamentos & A evolução de um tratamento deve ser registrada obrigatoriamente com data, cuidador responsável e observações. \\ \hline
RN06 & Tratamentos & Prescrições médicas devem ser anexadas em formato válido (PDF, JPG, PNG, etc.), respeitando o tamanho máximo definido para cada tipo de arquivo. \\ \hline
RN07 & Segurança & O armazenamento de todos os dados relacionados a um usuário, incluindo backups, devem ser realizados em ambientes seguros, hospedados em servidores certificados e com criptografia válida. \\ \hline
RN08 & Privacidade & O sistema deve solicitar consentimento explícito do usuário para a coleta e o armazenamento de dados sensíveis. \\ \hline
\end{tabularx}
\fonte{Autores.}
>>>>>>> origin/main
\end{quadro}
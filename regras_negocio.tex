As regras de negócio são fundamentais para garantir o correto funcionamento do sistema, assegurando que as operações atendam aos requisitos legais e funcionais. A seguir, apresentamos uma tabela com as principais regras de negócio definidas para o sistema de gestão de dependentes.
\renewcommand{\arraystretch}{1.5} % 1.5 = 50% mais alto que o normal

\begin{quadro}
\caption{Regras de Negócio do MyMed}
\begin{tabularx}{\textwidth}{|l|l|X|}
\hline
\textbf{Código} & \textbf{Categoria} & \textbf{Descrição} \\ \hline
RN01 & Cadastro & Apenas usuários com permissão de cuidador ou administrador podem cadastrar pacientes. \\ \hline
RN02 & Segurança & A exclusão de qualquer registro relacionado a um usuário, como o perfil de um paciente ou seu próprio perfil, deve arquivar todo o histórico sem remoção definitiva do banco de dados durante 15 dias, conforme prazo de resposta da LGPD. \\ \hline
RN03 & Tratamentos & Um plano de tratamento só pode ser criado se houver, no momento do registro, ao menos um diagnóstico relacionado. \\ \hline
RN04 & Tratamentos & A conclusão ou o cancelamento de um tratamento só podem ser realizados por cuidadores com permissão e devem ser registrados em log. \\ \hline
RN05 & Tratamentos & A evolução de um tratamento deve ser registrada obrigatoriamente com data, cuidador responsável e observações. \\ \hline
RN06 & Tratamentos & Prescrições médicas devem ser anexadas em formato válido (PDF, JPG, PNG, etc.), respeitando o tamanho máximo definido para cada tipo de arquivo. \\ \hline
RN07 & Segurança & O armazenamento de todos os dados relacionados a um usuário, incluindo backups, devem ser realizados em ambientes seguros, hospedados em servidores certificados e com criptografia válida. \\ \hline
RN08 & Privacidade & O sistema deve solicitar consentimento explícito do usuário para a coleta e o armazenamento de dados sensíveis. \\ \hline
\end{tabularx}
\fonte{Autores.}
\end{quadro}
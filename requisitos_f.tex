Os requisitos funcionais são essenciais para definir as funcionalidades que o sistema deve oferecer, garantindo que atenda às necessidades dos usuários e aos objetivos do negócio. A seguir, apresentamos uma tabela com os principais requisitos funcionais definidos para o sistema de gestão de dependentes.

\begin{longtblr}[
  label = requisitos_f,
  entry = none,
  caption = {Requisitos Funcionais},
  note = {Fonte: Autores.},
]{
  vline{1-4} = {-}{},
  hline{-} = {1-3}{},
}
Código & Categoria    & Descrição                                                                                       &  \\
RF01   & Cadastro     & {Cadastrar paciente: dados pessoais \\(nome, CPF, data de nascimento, endereço, contato, etc.)} &  \\
RF02   & Paciente     & {Visualizar perfil do paciente: \\histórico de atendimentos e tratamentos}                      &  \\
RF03   & Paciente     & Editar e atualizar dados do paciente                                                            &  \\
RF04   & Cadastro     & {Excluir cadastro de paciente: \\mas mantendo o histórico arquivado}                            &  \\
RF05   & Consulta     & {Agendar nova consulta: \\paciente, profissional, data, hora e local}                           &  \\
RF06   & Consulta     & {Listar consultas agendadas:\\~com filtros por data, profissional ou paciente}                  &  \\
RF07   & Consulta     & {Editar ou cancelar consulta: \\antes da data marcada}                                          &  \\
RF08   & Atendimento  & {Registrar atendimento: \\diagnóstico, conduta, recomendações, etc}                             &  \\
RF09   & Notificação  & {Emitir alertas ou notificações:\\~consultas futuras}                                           &  \\
RF10   & Tratamento   & {Criar plano de tratamento: \\associado a um paciente e diagnóstico}                            &  \\
RF11   & Tratamento   & {Listar tratamentos em andamento,\\concluídos ou cancelados (tipo kanban)}                      &  \\
RF12   & Tratamento   & {Registrar evolução do tratamento: \\observações por etapa ou sessão}                           &  \\
RF13   & Tratamento   & {Anexar prescrições médicas: \\laudos, imagens ou documentos ao tratamento}                     &  \\
RF14   & Relatório    & {Gerar relatórios de acompanhamento:\\paciente, profissional ou período}                        &  \\
RF15   & Histórico    & {Visualizar histórico de consultas: \\evolução clínica}                                         &  \\
RF16   & Exportação   & {Exportar dados: \\PDF, Excel ou outro formato}                                                 &  \\
RF17   & Autenticação & {Autenticar usuários: \\login e senha}                                                          &  \\
RF18   & Permissão    & {Gerenciar permissões: \\admin, profissional de saúde, recepcionista, etc}                      &  \\
RF19   & Log          & {Registrar logs de acesso:\\operações críticas (como edições e exclusões)}                      &  \\
RF20   & Pesquisa     & {Pesquisar pacientes, consultas e tratamentos:\\múltiplos critérios}                            &  \\
RF21   & Filtro       & Aplicar filtros e ordenações nas pesquisas                                                      &  
\end{longtblr}
\fonte{Autores.}
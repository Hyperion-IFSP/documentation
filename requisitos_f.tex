\begin{quadro}
\caption{Requisitos Funcionais do MyMed}
\begin{tabularx}{\textwidth}{|l|l|X|}
\hline
\textbf{Código} & \textbf{Categoria} & \textbf{Descrição} \\ \hline
RF01 & Paciente & O sistema deve manter pacientes e seus dados pessoais através de operações de criação, visualização, edição e remoção. \\ \hline
RF02 & Consulta & O sistema deve manter consultas e suas informações através de operações de criação, visualização, edição e remoção. \\ \hline
RF03 & Consulta & O sistema deve permitir o registro de relatórios pós-consultas e eventos na agenda, como posologias, recomendações ou observações. \\ \hline
RF04 & Notificações & O sistema deve emitir alertas ou notificações para eventos futuros, como o horário de uma dose de uma medicação ou consultas próximas. \\ \hline
RF05 & Tratamentos & O sistema deve manter tratamentos e suas informações através de operações de criação, visualização, edição e remoção. \\ \hline
RF06 & Tratamentos & O sistema deve permitir a adição de informações adicionais aos tratamentos como observações, sessões e documentos. \\ \hline
RF07 & Relatórios & O sistema deve permitir gerar relatórios de acompanhamento, de acordo com as informações do paciente e filtros como períodos e tratamentos. \\ \hline
RF08 & Histórico & O sistema deve permitir a visualização do histórico de consultas a fim de acompanhar a evolução clínica do paciente. \\ \hline
RF09 & Exportação & O sistema deve permitir a exportação de relatórios e gráficos em formatos como Excel e PDF, disponibilizando opções adequadas definidas a partir do conteúdo que irá ser exportado. \\ \hline
RF10 & Log & O sistema deve registrar logs de acesso e operações críticas (como edições ou exclusões). \\ \hline
RF11 & Filtros & O sistema deve aplicar corretamente filtros e ordenações determinados pelo usuário em determinadas ações. \\ \hline
RF12 & Validação & Campos obrigatórios devem ser validados antes de salvar qualquer registro. \\ \hline
RF13 & Validação & Dados médicos (como glicemia e pressão arterial) devem ser validados para garantir que estão dentro de intervalos adequados. \\ \hline
\end{tabularx}
\fonte{Autores.}
\end{quadro}
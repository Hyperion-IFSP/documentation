%% abtex2-modelo-artigo.tex, v-1.9.7 laurocesar
%% Copyright 2012-2018 by abnTeX2 group at http://www.abntex.net.br/ 
%%
%% This work may be distributed and/or modified under the
%% conditions of the LaTeX Project Public License, either version 1.3
%% of this license or (at your option) any later version.
%% The latest version of this license is in
%%   http://www.latex-project.org/lppl.txt
%% and version 1.3 or later is part of all distributions of LaTeX
%% version 2005/12/01 or later.
%%
%% This work has the LPPL maintenance status `maintained'.
%% 
%% The Current Maintainer of this work is the abnTeX2 team, led
%% by Lauro César Araujo. Further information are available on 
%% http://www.abntex.net.br/
%%
% ------------------------------------------------------------------------
% ------------------------------------------------------------------------
% abnTeX2: Modelo de Artigo Acadêmico em conformidade com
% ABNT NBR 6022:2018: Informação e documentação - Artigo em publicação 
% periódica científica - Apresentação
% ------------------------------------------------------------------------
% ------------------------------------------------------------------------

\documentclass[
	% -- opções da classe memoir --
	article,			% indica que é um artigo acadêmico
	12pt,				% tamanho da fonte
	oneside,			% para impressão apenas no recto. Oposto a twoside
	a4paper,			% tamanho do papel. 
	% -- opções da classe abntex2 --
	%chapter=TITLE,		% títulos de capítulos convertidos em letras maiúsculas
	%section=TITLE,		% títulos de seções convertidos em letras maiúsculas
	%subsection=TITLE,	% títulos de subseções convertidos em letras maiúsculas
	%subsubsection=TITLE % títulos de subsubseções convertidos em letras maiúsculas
	% -- opções do pacote babel --
    BIBLATEX,           % indica para utilizar BIBLATEX em vez do abntex2cite
	english,			% idioma adicional para hifenização
	brazil,				% o último idioma é o principal do documento
	sumario=tradicional
	]{abntex2}


% ---
% PACOTES
% ---

% ---
% Pacotes fundamentais 
% ---
\usepackage{sbc-template}
\usepackage{times}			% Usa a fonte Times new Roman
\usepackage[T1]{fontenc}		% Selecao de codigos de fonte.
\usepackage[utf8]{inputenc}		% Codificacao do documento (conversão automática dos acentos)
\usepackage{indentfirst}		% Indenta o primeiro parágrafo de cada seção.
\usepackage{nomencl} 			% Lista de simbolos
\usepackage{color}				% Controle das cores
\usepackage{graphicx}			% Inclusão de gráficos
\usepackage{microtype} 			% para melhorias de justificação

% ---
		
% ---
% Pacotes adicionais, usados apenas no âmbito do Modelo Canônico do abnteX2
% ---
\usepackage{lipsum}				% para geração de dummy text
% ---
		
% ---
% Pacotes de citações
% ---
\usepackage[brazilian,hyperpageref]{backref}	 % Paginas com as citações na bibl
\usepackage[alf]{abntex2cite}	% Citações padrão ABNT
% ---

% Pacotes instalados por nós alunos
\usepackage{xurl}

% ---
% Configurações do pacote backref
% Usado sem a opção hyperpageref de backref
\renewcommand{\backrefpagesname}{Citado na(s) página(s):~}
% Texto padrão antes do número das páginas
\renewcommand{\backref}{}
% Define os textos da citação
\renewcommand*{\backrefalt}[4]{
	\ifcase #1 %
		Nenhuma citação no texto.%
	\or
		Citado na página #2.%
	\else
		%Citado #1 vezes nas páginas #2.%
        Citado nas páginas #2.%
	\fi}%
% ---

% --- Informações de dados para CAPA e FOLHA DE ROSTO ---

\newcommand\nomeprojeto{MyMed}

\title{Documento de Arquitetura de Software de \nomeprojeto}

\author{Arthur Augusto Lessa Ferreira\inst{1}, Fernando Freitas de Lira\inst{1}, Isabella Pantolfo Melo\inst{1},\\ Henriquy Dias Terto Alves\inst{1}, Lucas da Conceição Silva Moura\inst{1},\\ Mateus Armando Carrara de Mendonça\inst{1} }


\address{
    Instituto Federal de Educação, Ciência e Tecnologia de São Paulo - Campus\\ São Paulo (IFSP) - Rua Pedro Vicente, 625 - Bloco C
}


% ---

% ---
% Configurações de aparência do PDF final

% alterando o aspecto da cor azul
\definecolor{blue}{RGB}{41,5,195}

% informações do PDF
\makeatletter
\hypersetup{
     	%pagebackref=true,
		pdftitle={\@title}, 
		pdfauthor={\@author},
    	pdfsubject={Modelo de artigo científico com abnTeX2},
	    pdfcreator={LaTeX with abnTeX2},
		pdfkeywords={abnt}{latex}{abntex}{abntex2}{atigo científico}, 
		colorlinks=true,       		% false: boxed links; true: colored links
    	linkcolor=blue,          	% color of internal links
    	citecolor=blue,        		% color of links to bibliography
    	filecolor=magenta,      		% color of file links
		urlcolor=blue,
		bookmarksdepth=4
}
\makeatother
% --- 

% ---
% compila o indice
% ---
\makeindex
% ---

% ---
% Altera as margens padrões
% ---
\setlrmarginsandblock{3cm}{3cm}{*} %{ESQUERDA}{DIREITA}
\setulmarginsandblock{3.5cm}{2.5cm}{*} %{SUPERIOR}{INFERIOR}
\checkandfixthelayout
% ---

% --- 
% Espaçamentos entre linhas e parágrafos 
% --- 

% O tamanho do parágrafo é dado por (está definido dentro do sbc-template.sty):
%\setlength{\parindent}{1.3cm}

% Controle do espaçamento entre um parágrafo e outro (está definido dentro do sbc-template.sty):
%\setlength{\parskip}{0.2cm}  % tente também \onelineskip

% Espaçamento simples
\SingleSpacing


% ----
% Início do             
% ----
\begin{document}

% Seleciona o idioma do documento (conforme pacotes do babel)
%\selectlanguage{english}
\selectlanguage{brazil}


% Retira espaço extra obsoleto entre as frases.
\frenchspacing 

% ----------------------------------------------------------
% ELEMENTOS PRÉ-TEXTUAIS
% ----------------------------------------------------------

%---
%
% Se desejar escrever o artigo em duas colunas, descomente a linha abaixo
% e a linha com o texto ``FIM DE ARTIGO EM DUAS COLUNAS''.
% \twocolumn[    		% INICIO DE ARTIGO EM DUAS COLUNAS
%
%---

% página de titulo principal (obrigatório)
\maketitle

% titulo em outro idioma (opcional)

\begin{abstract}
  The paper describes the current architecture of the software being refered as
  ``\nomeprojeto``, its main focus is to help pacients with a wide variety of
  medicine to remember their intake date and to monitor their use/stock.
  The content speaks of its concept and application, furthermore its
  data modelling and tools used.
\end{abstract}
     
\begin{resumo1} 
  Este documento descreve a arquitetura vigente do software designado como
  ``\nomeprojeto``, ele visa auxiliar pacientes com uma ampla gama de medicamentos
  a lembrar de seus horários de consumo e também a monitorar o uso/estoque deles.
  O conteúdo presente discorre sobre seu conceito e aplicação, além da
  modelagem de dados e ferramentas utilizadas.
\end{resumo1}



% ]  				% FIM DE ARTIGO EM DUAS COLUNAS
% ---

%\begin{center}\smaller
%\textbf{Data de submissão e aprovação}: elemento obrigatório. Indicar dia, mês e ano

%\textbf{Identificação e disponibilidade}: elemento opcional. Pode ser indicado o endereço eletrônico, DOI, suportes e outras informações relativas ao acesso.
%\end{center}

% ----------------------------------------------------------
% ELEMENTOS TEXTUAIS
% ----------------------------------------------------------
\textual

% ----------------------------------------------------------
% Introdução
% ----------------------------------------------------------
\section{Introdução}

A saúde dos dias de hoje é em sua maioria amparada, e muitas vezes serve de motivação para a tecnologia mais atual. Estudos mostram que esses serviços estão em constante evolução e a cada ano sendo mais acessíveis e utilizados por ambos grupos de médicos e de pacientes.

Médicos possuem sistemas de bulário eletrônico que são organizados por filtros por tratamento ou sintomas, de forma que eles tenham mais facilidade ao receitar um medicamento para o paciente; já o paciente recebe sua receita em seu celular por SMS, por e-mail ou até \textit{Whatsapp}.

Um grupo muito beneficiado por essas tecnologias é o de idosos com mais de 60 anos. Segundo \citeonline{clarisse2016}, com a idade avançada e capacidades motoras prejudicadas, idosos necessitam de aplicativos com interfaces simples e funcionalidades diretas para realizar tarefas do dia-a-dia ou até dentro do seu celular.

Aplicativos de assistência ao idoso são exemplos de um mercado promissor e de uma gama de usuários que buscam este serviço, como comunicação com um médico especializado,  serviços de ligação de emergência ou para melhor interagir com o celular.

Segundo a pesquisa de \citeonline{mariacristina2008}, idosos relatam dificuldades ao lembrar dos horários de consumo de medicamentos, e muitas vezes não conseguem seguir a prescrição médica. Um assunto sério que deve ser tratado com prioridade, pois o não cumprimento da dosagem correta pode levar a complicações de saúde.

\begin{citacao}
“A revisão bibliográfica de Silva, Schimidt e Silva (18) relata ainda que 40\% a 75\% dos idosos não tomam seus medicamentos em horários e quantidades prescritas.” \cite{clarisse2016}
\end{citacao}

O uso de aplicações para o auxílio geral de seus tratamentos é algo vital para estes grupos. É, também, uma ferramenta útil para pacientes não idosos portadores de doenças crônicas como doenças do coração, diabetes, hipertensão, entre outros. Segundo \citeonline{clarisse2016}, a longo prazo, poucos métodos de memorização própria se mantém efetivos, e muitas vezes são esquecidos.

Em outro caso, a maior parte da população brasileira não acompanha a carteirinha de vacinação própria ou a de seus filhos. Estudos do IBOPE Inteligência, a pedido da Pfizer, mostram que menos de 50\% da população verificam os programas de vacina em sua região e mantém sua vacinação em dia \footnote{\url{https://www.pfizer.com.br/noticias/releases/metade-dos-brasileiros-nao-observa-se-carteirinha-de-vacinacao-esta-em-diametade-dos-brasileiros-nao-observa-se-carteirinha-de-vacinacao-esta-em-dia}}.

O estudo relata também que cerca de 92\% da população pretende se vacinar no ano posterior ao da data da pesquisa, com a finalidade de se prevenir de doenças. Demonstra-se que há sim um interesse da população em prevenção, mas não possuem os métodos e informações necessárias para que a carteira vacinal esteja em dia.

Segundo \citeonline{lucas-cnn2023}, é uma questão de dificuldade de acesso a informações que dizem respeito à carteirinha. As mães de filhos pequenos relatam não conhecer os horários de funcionamento de locais de vacinação, ou até mesmo não saberem onde estão localizados. Além do desconhecimento, há medo e desconfiança de vacinas, muitas vezes motivados por \textit{fake news} ou familiares e amigos.

O problema que se visa solucionar é, então, a organização do consumo de remédios e outros tratamentos que necessitam de um horário, dosagem, ou planejamento prévio, além de datas de vacinação; ambas fornecidas por meio de lembretes criados pelo usuário. Irá auxiliar na checagem de estoque e o total de unidades disponíveis, que irá alertar a falta do remédio quando este estiver acabando.


\subsection{Objetivos}

O objetivo do projeto é criar uma plataforma que auxilie o usuário a manter seu tratamento em dia, seja ele de medicamentos, vacinas ou consultas. O sistema irá permitir que o usuário registre os medicamentos que consome, as vacinas que tomou e as consultas que fez, além de permitir o registro de novos medicamentos e vacinas.

Dessa forma, pretende-se registrar e alertar sobre os recursos de saúde recorrentes do paciente, a fim de motivá-lo a manter seu tratamento, consumo de medicações ou vacinas de acordo às recomendações médicas de acordo com as necessidades do usuário, contribuindo para sua saúde.

Para isso, procura-se realizar uma pesquisa de mercado de outras aplicações parecidas, afim de criar uma experiência única de usuário; com isso se torna viável o desenvolvimento de um sistema que, com uma interface simples e intuitiva, além de um sistema de lembretes que o auxilie a manter seu tratamento, torne automático uma tarefa crucial para o dia-a-dia.


\subsection{Justificativa}    

O consumo de um medicamento ou consultas de rotina podem se tornar algo supérfluo na rotina de um paciente que os realiza com frequência, podendo acarretar em um esquecimento de tais compromissos.

Com isso em mente, foi realizada uma pesquisa que apontou que 70\% dos participantes da pesquisa utilizam muito pouco de serviços tecnológicos de saúde, e dos que utilizam, relatam não corresponder às suas expectativas. Os resultados da pesquisa também apontaram que 62\% dos usuários têm grandes dificuldades em lembrar de datas, tanto de consultas come de vacinas periódicas, e 81\% relataram terem grandes problemas em lembrar dos horários de tomar seus medicamentos.

Este documento, portanto, demonstra a necessidade de tal sistema, do qual apelidamos de ``\nomeprojeto``. Busca-se auxiliar os usuários que necessitam de um melhor gerenciamento de seus tratamentos, sejam eles medicamentos, consultas ou até programas de vacinação; para manter sua saúde bem condicionada e supervisionada.

\section{Revisão da Literatura (ou Revisão Bibliográfica)}

Especial atenção ao que este capítulo deve conter:
    \begin{citacao}
    "Revisão bibliográfica, conforme já comentado, não produz conhecimento novo, mas apenas supre as
    deficiências de conhecimento que o pesquisador tem em uma determinada área. Portanto, ela deve ser muito
    bem planejada e conduzida.
    (...)
    Quando se faz uma pesquisa em que alguma técnica de computação é aplicada a alguma outra área do
    conhecimento, é necessário que se faça a revisão bibliográfica sobre a técnica em si, sobre a área de aplicação e,
    mais do que tudo, sobre as aplicações que já foram tentadas com essa técnica ou com técnicas semelhantes na
    mesma área ou em áreas equivalentes. Exemplificando, um aluno pretende desenvolver um sistema
    multiagentes para auxiliar controladores de voo. Esse aluno deve conhecer profundamente os sistemas
    multiagentes e deverá conhecer também os problemas que os controladores de voo enfrentam para exercer sua
    profissão. Porém, ele não deve pensar, como algumas vezes acontece, que essa é a primeira vez que alguém vai
    tentar desenvolver um sistema multiagentes para esse tipo de aplicação."
    \cite{PESQUISA:RAUL}.
    \end{citacao}

Toda a revisão da literatura deve ser basear primordialmente em livros e artigos científicos ranqueados Qualis CAPES. De forma geral, todo parágrafo deve conter AO MENOS uma citação bibliográfica.

% ---
\subsection{Assunto 1}

Atenção!!! Para quem está usando o modelo desde antes de 09/08/2024, é preciso alterar o arquivo sbc-template.sty e comentar as linhas com os seguintes comandos ou muda4r o arquivo para o contido nesse modelo caso não tenha realizado nenhuma alteração:

\verb|\RequirePackage[bf,sf,footnotesize,indent]{caption2}|
\verb|\setlength{\captionmargin}{0.8cm}|
\verb|\renewcommand{\captionfont}{\sffamily\footnotesize\bfseries}|
\verb|\renewcommand{\captionlabeldelim}{.}|

\index{figuras}Figuras podem ser criadas diretamente em \LaTeX, como o exemplo da  \autoref{fig_circulo}, ou inseridas a partir de arquivos externos como a \autoref{fig_logo}, que é o Logotipo do IFSP.

%Aqui uma figura criada em código.
\begin{figure}[htb]
	\caption{\label{fig_circulo}A delimitação do espaço}
	\begin{center}
	    \setlength{\unitlength}{5cm}
		\begin{picture}(1,1)
		\put(0,0){\line(0,1){1}}
		\put(0,0){\line(1,0){1}}
		\put(0,0){\line(1,1){1}}
		\put(0,0){\line(1,2){.5}}
		\put(0,0){\line(1,3){.3333}}
		\put(0,0){\line(1,4){.25}}
		\put(0,0){\line(1,5){.2}}
		\put(0,0){\line(1,6){.1667}}
		\put(0,0){\line(2,1){1}}
		\put(0,0){\line(2,3){.6667}}
		\put(0,0){\line(2,5){.4}}
		\put(0,0){\line(3,1){1}}
		\put(0,0){\line(3,2){1}}
		\put(0,0){\line(3,4){.75}}
		\put(0,0){\line(3,5){.6}}
		\put(0,0){\line(4,1){1}}
		\put(0,0){\line(4,3){1}}
		\put(0,0){\line(4,5){.8}}
		\put(0,0){\line(5,1){1}}
		\put(0,0){\line(5,2){1}}
		\put(0,0){\line(5,3){1}}
		\put(0,0){\line(5,4){1}}
		\put(0,0){\line(5,6){.8333}}
		\put(0,0){\line(6,1){1}}
		\put(0,0){\line(6,5){1}}
		\end{picture}
	\end{center}
	\fonte{Modelo Canônico ABNTeX2.}
\end{figure}

Procure criar suas imagens e diagramas pensando na possibilidade de utilizar impressão em preto-e-branco ou escala de cinza. Isto é importante, principalmente quando se pretende publicar o trabalho, uma vez que a maioria das publicações são somente em preto-e-branco. Outro benefício é o custo de impressão, normalmente menor para páginas preto-e-branco em relação a páginas coloridas.

\begin{figure}[htb]
    \centering
	\caption{\label{fig_logo}Logotipo IFSP}
	\includegraphics{Figuras/logoIFSP.jpg}
	\fonte{IFSP}
\end{figure}

Lembrem-se que o \LaTeX vai posicionar a figura o mais perto o possível do local onde ela está sendo citada tentando não deixar espaços em branco. Eivtem forçar a posição da figura.


\subsection{Assunto 2}

Atenção!!! Para quem está usando o modelo desde antes de 20/08/2024, é preciso
alterar o arquivo sbc-template.sty e acrescentar as linhas contidas entre as tags marcadas com: "============= COPIAR PARA SEU STY - 19/08/2024 =============".

Quadros e Tabelas são informações tabulares, mas Tabelas tem como objetivo apresentar números. A ‘norma’ 14724 \cite{NBR14724:2011} define a Tabela como sendo uma "forma não discursiva de apresentar informações das quais o dado numérico se destaca como informação central" e que devem seguir padronização do IBGE \cite{NBR14724:2011}. O IBGE padronizou a apresentações de dados tabulares em 1993 \cite{tabular-ibge}.

Informações adicionais sobre o de tabelas no {\LaTeX} podem ser obtidas em  \url{https://en.wikibooks.org/wiki/LaTeX/Tables}.

Antes de utilizar \index{longtable}\textbf{longtable} procure reorganizar o seu layout ou quebrar manualmente em múltiplos quadros / tabelas, pois isso ainda facilita a compreensão pelo leitor.

% https://biblioteca.ibge.gov.br/visualizacao/livros/liv23907.pdf

O \autoref{quadro_exemplo} é um exemplo de dados tabulares gerados em \LaTeX.

\begin{quadro}[htb]
\caption{\label{quadro_exemplo}Exemplo de quadro}
\begin{tabular}{|c|c|c|c|}
	\hline
	\textbf{Pessoa} & \textbf{Idade} & \textbf{Peso} & \textbf{Altura} \\ \hline
	Marcos & 26    & 68   & 178    \\ \hline
	Ivone  & 22    & 57   & 162    \\ \hline
	...    & ...   & ...  & ...    \\ \hline
	Sueli  & 40    & 65   & 153    \\ \hline
\end{tabular}
\fonte{Autor.}
\end{quadro}

Já a \autoref{tab-exemplo} foi criada conforme o padrão \citeonline{tabular-ibge} requerido pelas normas da ABNT para documentos técnicos e acadêmicos. Observe que não existem bordas laterais e nem linhas separadoras em uma Tabela e as colunas numéricas tem alinhamento à direita. 

\begin{table}[htb]
\centering
\caption{Métricas de desenvolvimento}
\label{tab-exemplo}
\begin{tabular}{p{2.6cm}rrr}
    \hline
    \textbf{Item} & \textbf{Janeiro} & \textbf{Fevereiro} & \textbf{Março} \\
    \hline
    \hline
    Classes & 2  & 10 & 20  \\
    Linhas & 100  & 250 & 543 \\
    \hline
\end{tabular}
\fonte{Os autores.}
\end{table}

Para facilitar a criação de tabelas e quadros existem algumas ferramentas como o Tables Generator \url{http://www.tablesgenerator.com/latex_tables} que permite a criação de forma visual gerando o código \LaTeX\ correspondente. E o site \url{https://www.latex-tables.com/} permite converter planilhas em código \LaTeX.

\subsection{Assunto 3}
\lipsum[1]
\subsection{Assunto X}
\lipsum[1]


\section{Métodos de Pesquisa OU Materiais e métodos}

Segundo \citeonline{PESQUISA:DEMO}, metodologia significa, “na origem do termo, estudo dos caminhos, dos instrumentos usados para se fazer ciência”.

Completando a linha de raciocínio, o autor acrescenta:

    \begin{citacao}
    “Alguns entendem por pesquisa o trabalho de coletar dados, sistematizá-los e, a partir daí fazer uma descrição da real-dade. Outros, fixam-se no patamar teórico e entendem por pesquisa o estudo e a produção de quadros teóricos de referência que estaria na origem da explicação da realidade. Descrever restringe-se a constatar o que já existe. Explicar corresponde a desvendar por que existe. Outros mais acreditam que pesquisar inclui teoria e prática. Porque compreender a realidade e nela intervir formam um todo só, tornando-se vício oportunista ficar apenas na constatação descritiva ou apenas na especulação teórica.”
    \lipsum[5] \cite{PESQUISA:DEMO}.
    \end{citacao}



As seções a seguir são sugestões do que pode estar na metodologia. Conversem com o(s) professor(es) em busca de ajuda para definir quais as seções mais adequadas para cada trabalho.

\subsection{Tipo de Pesquisa}
\lipsum[1]

\subsection{Plano Amostral (se Pesquisa Quantitativa)}
\lipsum[1]

\subsection{Instrumento de Pesquisa e Escalas Utilizadas (Escalas se Pesquisa Quantitativa)}
\lipsum[1]

\subsection{Coleta de Dados}
\lipsum[1]

\subsection{Análise de Dados}
\lipsum[1]

\subsection{Materiais}
Para desenvolver uma aplicação web, faz-se necessário o uso de diversos materiais, os quais vão desde uma linguagem de programação específica até um navegador qualquer, dessa forma, serão listadas a seguir todas as ferramentas que serão utilizadas na elaboração do projeto.
	
 \subsection{Métodos}
Os métodos, modo como aplicamos as ferramentas no desenvolvimento, deixa claro como será feito todo o processo de criação do sistema.

\subsection{Embasamento Inicial}
\lipsum[1]

\subsection{Desenvolvimento do Software}
\lipsum[1]

\subsection{Metodologias de Desenvolvimento}
\lipsum[1]

\section{Desenvolvimento}

\subsection{Equipe}
\lipsum[1]

\subsection{Requisitos}
\lipsum[1]

\subsection{Modelagem}
\lipsum[1]

\subsection{Prototipagem}
\lipsum[1]

\section{POC}

A Prova de Conceito (\textit{Proof of Concept} (PoC)) que deve demonstrar a aderência das tecnologias escolhidas com a aplicação que deve ser desenvolvida. Essa prova de conceito deve demonstrar a comunicação desde o usuário até a base de dados e utilizar de forma simples as tecnologias escolhidas para demonstrar que
elas funcionam para o objetivo desejado.

\section{MVP}

O termo MVP foi popularizado por  \citeonline{ries2011lean}, onde ele descreve o conceito como segue:

"O MVP é o menor conjunto de recursos que permite que o empreendedor comece o processo de aprendizado com o mínimo de esforço e o máximo de aprendizado validado sobre os clientes."

Outro autor importante na área, \citeonline{blank2013startup}, define o MVP como:

"Uma ferramenta para testar hipóteses de negócios e iniciar o aprendizado, coletando o máximo de informações validadas sobre os clientes com o menor esforço possível."


% ---
% Finaliza a parte no bookmark do PDF, para que se inicie o bookmark na raiz
% ---
\bookmarksetup{startatroot}% 
% ---

% ---
% Conclusão
% ---
\section{Considerações finais}


De acordo com \citeonline{severino2016metodologia}, na seção de considerações finais o autor tem a oportunidade de fazer uma síntese dos principais pontos abordados e apresentar suas considerações finais sobre o assunto. Embora não haja uma estrutura fixa, existem algumas diretrizes comuns para escrever essa seção.

A seguir, algumas orientações gerais, para complementar a explicação:

1. Recapitule os principais pontos: Na seção de considerações finais, você pode revisitar os principais pontos discutidos ao longo do trabalho e resumir os resultados obtidos. É uma oportunidade para destacar a relevância do estudo e como ele contribui para o conhecimento existente.

2. Discuta as implicações dos resultados: Nessa seção, você pode discutir as implicações práticas e teóricas dos resultados do seu trabalho. 

3. Faça uma reflexão crítica: Use a seção de considerações finais para fazer uma reflexão crítica sobre as limitações do estudo e possíveis viéses. Discuta as dificuldades encontradas, bem como eventuais lacunas de conhecimento que podem ser exploradas por estudos futuros.

4. Encerre de forma concisa e impactante: Finalize a seção de considerações finais com uma frase ou parágrafo que resuma as principais conclusões e destaque a importância do estudo. É uma oportunidade para deixar uma impressão duradoura nos leitores.

Além do exposto acima, colocamos aqui uma outra possibilidade de estrutura para o documento:

\begin{enumerate}
\item Introdução
1.1. Objetivo
\item Concepção Inicial
\item Trabalhos Correlatos
    \begin{enumerate}
        \item Trabalho 1
        \item Trabalho 2
        \item Trabalho 3
        \item Trabalho X
    \end{enumerate}
\item Referencial Teórico
\item Materiais e métodos
\item Modelagem do Sistema
    \begin{enumerate}
        \item Diagrama de Casos de Uso
        \item Diagrama de Tabelas Relacionais
        \item Diagrama Entidade-Relacionamento
    \end{enumerate}
\item Funcionalidades
\item Considerações Finais
\end{enumerate}
Referências



% ----------------------------------------------------------
% ELEMENTOS PÓS-TEXTUAIS
% ----------------------------------------------------------
\postextual

% ----------------------------------------------------------
% Referências bibliográficas
% ----------------------------------------------------------

% ----------------------------------------------------------
% Glossário
% ----------------------------------------------------------
%
% Há diversas soluções prontas para glossário em LaTeX. 
% Consulte o manual do abnTeX2 para obter sugestões.
%
%\glossary
% ----------------------------------------------------------
% Apêndices
% ----------------------------------------------------------

\bibliography{referencias}
% ---
% Inicia os apêndices
% ---
\newpage
\begin{apendicesenv}

% ----------------------------------------------------------
\chapter{Nullam elementum urna vel imperdiet sodales elit ipsum pharetra ligula
ac pretium ante justo a nulla curabitur tristique arcu eu metus}
% ----------------------------------------------------------
Apêndices e anexos são materiais complementares ao texto que só devem ser incluídos quando forem imprescindíveis à compreensão deste.

Apêndices são textos elaborados pelo autor a fim de complementar sua argumentação.

Os apêndices devem aparecer após as referências, e os anexos, após os apêndices.

\end{apendicesenv}
% ---

% ----------------------------------------------------------
% Anexos
% ----------------------------------------------------------
\cftinserthook{toc}{AAA}
% ---
% Inicia os anexos
% ---
%\anexos
\newpage
\begin{anexosenv}

% ---
\chapter{Cras non urna sed feugiat cum sociis natoque penatibus et magnis dis
parturient montes nascetur ridiculus mus}
% ---

Anexos são os documentos não elaborados pelo autor, que servem de fundamentação, comprovação ou ilustração, como mapas, leis, estatutos etc.

Os apêndices devem aparecer após as referências, e os anexos, após os apêndices.

\end{anexosenv}

\end{document}

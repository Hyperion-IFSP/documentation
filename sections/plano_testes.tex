\setcounter{section}{0}
\section{Plano de Testes}

O quadro apresenta o plano de testes do sistema, detalhando os principais objetos de teste e os critérios de sucesso para cada caso. Entre os itens testados estão a manutenção de usuários e admins, gestão de informações adicionais ao perfil, registro e consulta de pacientes, notificações de tratamento, pesquisa de farmácias e medicamentos, além do controle de registros de saúde como pressão arterial, glicose e adesão a medicamentos. Cada teste busca garantir a correta criação, edição, exclusão e visualização dos dados no sistema, assegurando o funcionamento esperado das funcionalidades essenciais para o gerenciamento de pacientes e seus tratamentos.

\begin{quadro}
    \caption{\label{quadro_plano_testes}Plano de Testes do Sistema}
    \begin{tabular}{|l|p{6cm}|p{5cm}|p{5cm}|}
        \hline
        \textbf{Código} & \textbf{Objeto de Teste} & \textbf{Sucesso} & \textbf{Erro} \\ \hline
        T01 & Manter usuário/admin & Criação, edição ou exclusão de usuário ou admin no banco de dados & A conexão com o banco de dados pode falhar \\ \hline
        T02 & Manter informações adicionais ao perfil de usuário & Adição, edição ou exclusão de número de telefone, foto de perfil e endereço & A conexão com o banco de dados pode falhar; a foto de perfil pode apresentar erros de compressão \\ \hline
        T03 & Manter paciente & Criação, edição ou exclusão de paciente no banco de dados & A conexão com o banco de dados pode falhar \\ \hline
        T04 & Enviar ao banco consulta de paciente & Criação de consulta no banco de dados & A conexão com o banco de dados pode falhar \\ \hline
        T05 & Visualizar consultas por dia/mês de paciente & Consultas organizadas pelo filtro de dia ou mês & A conexão com o banco de dados pode falhar; o conteúdo pode aparecer de forma errada \\ \hline
        T06 & Adicionar medicamento a um tratamento de paciente & Paciente consta que possui tratamento com aquele medicamento & A conexão com o banco de dados pode falhar \\ \hline
        T07 & Receber notificação de tratamento de paciente & Visualizar notificação de tratamento no horário que foi inserido & A notificação pode não aparecer; a notificação pode aparecer no horário errado; a notificação pode aparecer pertinente a outro paciente \\ \hline
        T08 & Manter atendimento de consulta de paciente & Criação, edição ou exclusão de atendimento de consulta & A conexão com o banco de dados pode falhar \\ \hline
        T09 & Pesquisar locais de farmácia & Visualizar farmácias nas proximidades & A conexão com a API pode falhar; o GPS do usuário pode estar desligado \\ \hline
        T10 & Pesquisar medicamentos de um determinado tratamento a venda & Visualizar a venda de medicamentos utilizados em tratamentos & A conexão com a API pode falhar \\ \hline
        T11 & Manter registro de pressão diastólica e sistólica de paciente & Criação, edição ou exclusão de gráfico de pressão cardiovascular & O gráfico pode apresentar erros em sua exibição; informações podem ser registradas erroneamente \\ \hline
        T12 & Manter registro de nível de glicose de paciente & Criação, edição ou exclusão de gráfico de nível glicose & O gráfico pode apresentar erros em sua exibição; informações podem ser registradas erroneamente \\ \hline
        T13 & Visualizar registro de adesão a medicamentos de paciente & Visualizar taxa de adesão de paciente a medicamentos & O gráfico pode apresentar erros em sua exibição; informações podem ser registradas automaticamente de forma errada \\ \hline
        T14 & Manter intercorrências de paciente & Criação, edição ou exclusão de intercorrências de paciente & O gráfico pode apresentar erros em sua exibição; informações podem ser registradas erroneamente \\ \hline
    \end{tabular}
    \fonte{Autores.}
\end{quadro}

% \pagebreak

A prototipagem foi realizada com o objetivo de criar uma interface amigável e funcional para os usuários do sistema. Foram desenvolvidas telas para diferentes funcionalidades, permitindo uma interação intuitiva e eficiente.

\subsection{Consultas}

As telas de consulta foram desenvolvidas para permitir que o usuário visualize e gerencie suas consultas médicas.

A tela de avaliação de consulta, apresentada na \autoref{avaliacao_consulta}, permite ao usuário registrar o que aconteceu no encontro, anexar documentos importantes como receitas e atestados, e adicionar observações relevantes. O layout é limpo, com campos bem definidos para especialidade, data, local e um espaço para anotações, facilitando o registro detalhado. A tela complementar, vista na \autoref{avaliacao_pos_consulta_2}, oferece uma interface para anexar arquivos, como fotos de receitas ou resultados de exames.

\begin{figure}[!htbp]
	\centering
	\includegraphics[width=0.4\linewidth]{assets/prototipo/Avaliação Pos-consulta.png}
	\caption{Avaliação de Consulta}
	\fonte{Autores.}
	\label{avaliacao_consulta}
\end{figure}
A tela de nova consulta, conforme a \autoref{nova_consulta}, foi desenhada para ser direta, com campos para especialidade, data, hora e local. O design inclui um seletor de data e hora integrado e botões claros para salvar ou cancelar, tornando o agendamento de um novo compromisso uma tarefa rápida e sem complicações.

\begin{figure}[!htbp]
	\centering
	\includegraphics[width=0.4\linewidth]{assets/prototipo/Avaliação Pos-consulta (2).png}
	\caption{Avaliação de Consulta - Anexos}
	\fonte{Autores.}
	\label{avaliacao_pos_consulta_2}
\end{figure}

\begin{figure}[!htbp]
	\centering
	\includegraphics[width=0.4\linewidth]{assets/prototipo/Agendamento - Consulta - Nome.png}
	\caption{Nova Consulta}
	\fonte{Autores.}
	\label{nova_consulta}
\end{figure}

\subsection{Calendário}

As telas de calendário foram projetadas para ajudar o usuário a visualizar e gerenciar suas consultas médicas de forma eficiente.

A \autoref{calendario} apresenta uma visão em formato de lista, exibindo todos os compromissos futuros de forma cronológica.

\begin{figure}[!htbp]
	\centering
	\includegraphics[width=0.4\linewidth]{assets/prototipo/Agenda.png}
	\caption{Calendário}
	\fonte{Autores.}
	\label{calendario}
\end{figure}

A primeira tela de consulta finalizada no calendário mostra um pop-up sobreposto ao dia, oferecendo opções para avaliar a consulta ou reagendá-la. Este design interativo facilita a gestão de compromissos passados diretamente na visualização do calendário. A segunda tela de consulta finalizada (\autoref{calendario_consulta_finalizada}) exibe os detalhes da consulta, como especialidade e horário, com um status visual de "Finalizada". O layout permite que o usuário acesse rapidamente as informações do compromisso que já ocorreu.


% \begin{figure}[!htbp]
% 	\centering
% 	\includegraphics[width=0.6\linewidth]{assets/prototipo/calendario-consulta-finalizada-1.png}
% 	\caption{Calendário - Consulta Finalizada 1}
% 	\fonte{Autores.}
% 	\label{calendario_consulta_finalizada_1}
% \end{figure}

\begin{figure}[!htbp]
	\centering
	\includegraphics[width=0.4\linewidth]{assets/prototipo/Agendamento - Consulta - Adicionar na agenda.png}
	\caption{Calendário - Consulta Finalizada}
	\fonte{Autores.}
	\label{calendario_consulta_finalizada}
\end{figure}

\subsection{Agendamento}

O fluxo de agendamento de consultas foi desenhado para ser sequencial e intuitivo. O processo começa com a seleção do nome do paciente, como visto nas telas \autoref{agendamento_nome_1}, \autoref{agendamento_nome_2}, \autoref{agendamento_nome_3} e \autoref{agendamento_nome_4}, que demonstram a busca e seleção do dependente. Em seguida, o usuário define o horário da consulta, com interfaces claras para a escolha da hora e minutos, conforme ilustrado em \autoref{agendamento_horario} e \autoref{agendamento_horario_1}. Por fim, a tela \autoref{agendamento_add_agenda} oferece a opção de adicionar o compromisso à agenda do dispositivo, integrando o aplicativo à rotina do usuário.

\begin{figure}[!htbp]
	\centering
	\includegraphics[width=0.4\linewidth]{assets/prototipo/Agendamento - Consulta - Nome (1).png}
	\caption{Agendamento de Consulta - Seleção de Nome 1}
	\fonte{Autores.}
	\label{agendamento_nome_1}
\end{figure}

\begin{figure}[!htbp]
	\centering
	\includegraphics[width=0.4\linewidth]{assets/prototipo/Agendamento - Consulta - Nome (2).png}
	\caption{Agendamento de Consulta - Seleção de Nome 2}
	\fonte{Autores.}
	\label{agendamento_nome_2}
\end{figure}

\begin{figure}[!htbp]
	\centering
	\includegraphics[width=0.6\linewidth]{assets/prototipo/Agendamento - Consulta - Nome (3).png}
	\caption{Agendamento de Consulta - Seleção de Nome 3}
	\fonte{Autores.}
	\label{agendamento_nome_3}
\end{figure}

\begin{figure}[!htbp]
	\centering
	\includegraphics[width=0.6\linewidth]{assets/prototipo/Agendamento - Consulta - Nome (4).png}
	\caption{Agendamento de Consulta - Seleção de Nome 4}
	\fonte{Autores.}
	\label{agendamento_nome_4}
\end{figure}

\begin{figure}[!htbp]
	\centering
	\includegraphics[width=0.6\linewidth]{assets/prototipo/Agendamento - Consulta - Horario.png}
	\caption{Agendamento de Consulta - Seleção de Horário}
	\fonte{Autores.}
	\label{agendamento_horario}
\end{figure}

\begin{figure}[!htbp]
	\centering
	\includegraphics[width=0.6\linewidth]{assets/prototipo/Agendamento - Consulta - Horario (1).png}
	\caption{Agendamento de Consulta - Seleção de Horário 2}
	\fonte{Autores.}
	\label{agendamento_horario_1}
\end{figure}

\begin{figure}[!htbp]
	\centering
	\includegraphics[width=0.6\linewidth]{assets/prototipo/Agendamento - Consulta - Adicionar na agenda.png}
	\caption{Agendamento de Consulta - Adicionar na Agenda}
	\fonte{Autores.}
	\label{agendamento_add_agenda}
\end{figure}

\subsection{Gráficos}

As telas de gráficos foram desenvolvidas para fornecer uma visualização clara e intuitiva dos dados de saúde do usuário.

A tela principal de gráficos, ilustrada na \autoref{graficos}, serve como um painel central, oferecendo acesso a diferentes visualizações de dados de saúde, como adesão a medicações, intercorrências, glicemia e pressão. O layout utiliza cartões distintos para cada tipo de gráfico, facilitando a navegação. A tela de relatórios (\autoref{relatorios}) complementa essa visão, permitindo a geração de documentos consolidados para compartilhamento com profissionais de saúde.

\begin{figure}[!htbp]
	\centering
	\includegraphics[width=0.6\linewidth]{assets/prototipo/Graficos.png}
	\caption{Gráficos}
	\fonte{Autores.}
	\label{graficos}
\end{figure}

O gráfico de adesão a medicações (\autoref{grafico_adesao_medicacoes}) utiliza um gráfico de pizza para mostrar a porcentagem de doses administradas corretamente, em atraso ou não administradas. O design visual e colorido permite uma compreensão rápida do quão bem o tratamento está sendo seguido.

\begin{figure}[!htbp]
	\centering
	\includegraphics[width=0.6\linewidth]{assets/prototipo/Relatorios.png}
	\caption{Relatórios}
	\fonte{Autores.}
	\label{relatorios}
\end{figure}

\begin{figure}[!htbp]
	\centering
	\includegraphics[width=0.6\linewidth]{assets/prototipo/Grafico - Adesão a Medicacoes.png}
	\caption{Gráfico - Adesão a Medicações}
	\fonte{Autores.}
	\label{grafico_adesao_medicacoes}
\end{figure}

O gráfico de glicemia, detalhado na \autoref{grafico_glicemia}, apresenta as medições de açúcar no sangue ao longo do tempo em um gráfico de linhas. O layout inclui um seletor de período e destaca os valores com cores, facilitando a identificação de tendências.

\begin{figure}[!htbp]
	\centering
	\includegraphics[width=0.6\linewidth]{assets/prototipo/Grafico - Glicemia.png}
	\caption{Gráfico - Glicemia}
	\fonte{Autores.}
	\label{grafico_glicemia}
\end{figure}

O gráfico de intercorrências (\autoref{grafico_intercorrencias}) exibe a frequência de eventos adversos em um gráfico de barras, permitindo ao cuidador visualizar os tipos de ocorrências e suas quantidades. O design é claro e ajuda a monitorar a saúde do paciente.

\begin{figure}[!htbp]
	\centering
	\includegraphics[width=0.6\linewidth]{assets/prototipo/Grafico - Intercorrencias.png}
	\caption{Gráfico - Intercorrências}
	\fonte{Autores.}
	\label{grafico_intercorrencias}
\end{figure}

O gráfico de pressão arterial, como pode ser visto na \autoref{grafico_pressao}, mostra a evolução das medições sistólica e diastólica em um gráfico de linhas duplas. O layout permite filtrar por período e usa cores distintas para cada medição, tornando a análise cardiovascular simples e visual.

\begin{figure}[!htbp]
	\centering
	\includegraphics[width=0.6\linewidth]{assets/prototipo/Grafico - Pressao.png}
	\caption{Gráfico - Pressão}
	\fonte{Autores.}
	\label{grafico_pressao}
\end{figure}

\subsection{Configurações e Perfil}

As telas de login e perfil foram projetadas para garantir uma experiência de usuário segura e personalizada.

A tela de login, apresentada na \autoref{login}, possui um design minimalista, com campos para e-mail e senha, além de opções para login social (Google e Facebook) e um link para recuperação de senha. O layout é focado na simplicidade para garantir um acesso rápido e sem atritos ao sistema.

\begin{figure}[!htbp]
	\centering
	\includegraphics[width=1.0\linewidth]{assets/prototipo/Login.png}
	\caption{Login}
	\fonte{Autores.}
	\label{login}
\end{figure}

A tela de perfil, ilustrada na \autoref{perfil}, exibe as informações do usuário, como nome e e-mail, e oferece acesso a outras seções, como "Meus Dependentes" e "Configurações". O layout é organizado com um ícone de lápis que indica a possibilidade de edição, promovendo uma navegação intuitiva.

\begin{figure}[!htbp]
	\centering
	\includegraphics[width=0.8\linewidth]{assets/prototipo/Perfil de Usuário.png}
	\caption{Perfil}
	\fonte{Autores.}
	\label{perfil}
\end{figure}

A tela de edição de perfil (\autoref{perfil_edicao}) permite ao usuário atualizar suas informações pessoais, como nome, e-mail e senha. O layout utiliza campos de formulário claros e botões de ação bem visíveis para salvar as alterações ou cancelar a operação. A tela de configurações (\autoref{configuracoes}) centraliza opções importantes, como a gestão de contatos de emergência (\autoref{contato_emergencia}), permitindo um acesso rápido e organizado.

\begin{figure}[!htbp]
	\centering
	\includegraphics[width=0.8\linewidth]{assets/prototipo/Configuracoes.png}
	\caption{Configurações}
	\fonte{Autores.}
	\label{configuracoes}
\end{figure}

\begin{figure}[!htbp]
	\centering
	\includegraphics[width=0.6\linewidth]{assets/prototipo/Contato de Emergencia.png}
	\caption{Contato de Emergência}
	\fonte{Autores.}
	\label{contato_emergencia}
\end{figure}

\subsection{Gerenciamento de Pacientes e Tratamentos}

O sistema oferece um conjunto de telas para o gerenciamento completo de pacientes e seus tratamentos. A tela de pacientes (\autoref{pacientes}) lista todos os dependentes associados ao cuidador. A partir dela, é possível acessar o perfil detalhado de cada paciente (\autoref{perfil_paciente}), que inclui informações como a definição de dependência (\autoref{dependencia}), dados pessoais como sexo e idade (\autoref{sexo_idade}) e a foto de perfil (\autoref{foto}).

A gestão de tratamentos é centralizada na tela \autoref{tratamentos}, onde o cuidador pode visualizar e gerenciar as terapias em andamento. A tela de informações do tratamento (\autoref{info_tratamento}) exibe detalhes específicos de uma terapia, como medicamentos e dosagens. O fluxo de conclusão do cadastro é guiado pelas telas \autoref{finalizando} e \autoref{finalizando_1}, que confirmam o salvamento das informações.

\begin{figure}[!htbp]
	\centering
	\includegraphics[width=0.6\linewidth]{assets/prototipo/Pacientes.png}
	\caption{Pacientes}
	\fonte{Autores.}
	\label{pacientes}
\end{figure}

\begin{figure}[!htbp]
	\centering
	\includegraphics[width=0.6\linewidth]{assets/prototipo/Perfil de Paciente.png}
	\caption{Perfil de Paciente}
	\fonte{Autores.}
	\label{perfil_paciente}
\end{figure}

\begin{figure}[!htbp]
	\centering
	\includegraphics[width=0.6\linewidth]{assets/prototipo/Dependencia.png}
	\caption{Dependência}
	\fonte{Autores.}
	\label{dependencia}
\end{figure}

\begin{figure}[!htbp]
	\centering
	\includegraphics[width=0.6\linewidth]{assets/prototipo/Sexo e Idade.png}
	\caption{Sexo e Idade}
	\fonte{Autores.}
	\label{sexo_idade}
\end{figure}

\begin{figure}[!htbp]
	\centering
	\includegraphics[width=0.6\linewidth]{assets/prototipo/Nome.png}
	\caption{Perfil - Edição}
	\fonte{Autores.}
	\label{perfil_edicao}
\end{figure}

\begin{figure}[!htbp]
	\centering
	\includegraphics[width=0.6\linewidth]{assets/prototipo/Foto.png}
	\caption{Foto}
	\fonte{Autores.}
	\label{foto}
\end{figure}

\begin{figure}[!htbp]
	\centering
	\includegraphics[width=0.6\linewidth]{assets/prototipo/Tratamentos.png}
	\caption{Tratamentos}
	\fonte{Autores.}
	\label{tratamentos}
\end{figure}

\begin{figure}[!htbp]
	\centering
	\includegraphics[width=0.6\linewidth]{assets/prototipo/Informacoes do Tratamento.png}
	\caption{Informações do Tratamento}
	\fonte{Autores.}
	\label{info_tratamento}
\end{figure}

\begin{figure}[!htbp]
	\centering
	\includegraphics[width=0.6\linewidth]{assets/prototipo/Finalizando.png}
	\caption{Finalizando Cadastro}
	\fonte{Autores.}
	\label{finalizando}
\end{figure}

\begin{figure}[!htbp]
	\centering
	\includegraphics[width=0.6\linewidth]{assets/prototipo/Finalizando (1).png}
	\caption{Finalizando Cadastro 2}
	\fonte{Autores.}
	\label{finalizando_1}
\end{figure}

\subsection{Tela Inicial}

A tela inicial do aplicativo \nomeprojeto foi projetada para ser intuitiva e fácil de navegar, permitindo que os usuários acessem rapidamente as funcionalidades principais.

A tela inicial principal, como visto na \autoref{tela_inicial}, funciona como um painel de controle, exibindo as próximas consultas e medicamentos do dia. O layout é limpo, com um menu de navegação inferior para acesso rápido ao calendário, gráficos e perfil.

\begin{figure}[!htbp]
	\centering
	\includegraphics[width=1.0\linewidth]{assets/prototipo/Home.png}
	\caption{Tela Inicial}
	\fonte{Autores.}
	\label{tela_inicial}
\end{figure}

A tela com o menu dropdown de nomes expandido, conforme a \autoref{tela_inicial_dropdown_nomes}, mostra como o cuidador pode alternar rapidamente entre os perfis de diferentes dependentes. O design destaca o nome do paciente ativo e lista os outros de forma clara, otimizando a gestão de múltiplos pacientes.

\begin{figure}[!htbp]
	\centering
	\includegraphics[width=0.6\linewidth]{assets/prototipo/Pacientes.png}
	\caption{Tela Inicial - Dropdown Nomes}
	\fonte{Autores.}
	\label{tela_inicial_dropdown_nomes}
\end{figure}

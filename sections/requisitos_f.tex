Os requisitos funcionais do sistema são apresentados a seguir, servindo de descrição das principais funcionalidades que o sistema deve oferecer.

\renewcommand{\arraystretch}{1.5} % 1.5 = 50% mais alto que o normal

\begin{quadro}
    \caption{\label{quadro_requisitos_f}Requisitos Funcionais}
    \begin{tabular}{|c|c|p{10cm}|}
        \hline
        \textbf{Código} & \textbf{Categoria} & \textbf{Descrição} \\ \hline
        RF01   & Cadastro     & Cadastrar paciente: dados pessoais (nome, CPF, data de nascimento, endereço, contato, etc.) \\ \hline
        RF02   & Paciente     & Visualizar perfil do paciente: histórico de atendimentos e tratamentos \\ \hline
        RF03   & Paciente     & Editar e atualizar dados do paciente \\ \hline
        RF04   & Cadastro     & Excluir cadastro de paciente: mas mantendo o histórico arquivado \\ \hline
        RF05   & Consulta     & Agendar nova consulta: paciente, profissional, data, hora e local \\ \hline
        RF06   & Consulta     & Listar consultas agendadas: com filtros por data, profissional ou paciente \\ \hline
        RF07   & Consulta     & Editar ou cancelar consulta: antes da data marcada \\ \hline
        RF08   & Atendimento  & Registrar atendimento: diagnóstico, conduta, recomendações, etc. \\ \hline
        RF09   & Notificação  & Emitir alertas ou notificações: consultas futuras \\ \hline
        RF10   & Tratamento   & Criar plano de tratamento: associado a um paciente e diagnóstico \\ \hline
        RF11   & Tratamento   & Listar tratamentos em andamento, concluídos ou cancelados (tipo kanban) \\ \hline
        RF12   & Tratamento   & Registrar evolução do tratamento: observações por etapa ou sessão \\ \hline
        RF13   & Tratamento   & Anexar prescrições médicas: laudos, imagens ou documentos ao tratamento \\ \hline
        RF14   & Relatório    & Gerar relatórios de acompanhamento: paciente, profissional ou período \\ \hline
        RF15   & Histórico    & Visualizar histórico de consultas: evolução clínica \\ \hline
        RF16   & Exportação   & Exportar dados: PDF, Excel ou outro formato \\ \hline
        RF17   & Autenticação & Autenticar usuários: login e senha \\ \hline
        RF18   & Permissão    & Gerenciar permissões: admin, profissional de saúde, recepcionista, etc. \\ \hline
        RF19   & Log          & Registrar logs de acesso: operações críticas (como edições e exclusões) \\ \hline
        RF20   & Pesquisa     & Pesquisar pacientes, consultas e tratamentos: múltiplos critérios \\ \hline
        RF21   & Filtro       & Aplicar filtros e ordenações nas pesquisas \\ \hline
    \end{tabular}
    \fonte{Autores.}
\end{quadro}
